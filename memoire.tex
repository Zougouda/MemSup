% !TEX TS-program = pdflatex
% !TEX encoding = UTF-8 Unicode

% Example of the Memoir class, an alternative to the default LaTeX classes such as article and book, with many added features built into the class itself.

%\documentclass[12pt,a4paper]{memoir} % for a long document
\documentclass[12pt,a4paper,article]{memoir} % for a short document

\usepackage[utf8]{inputenc} % set input encoding to utf8
%LANGUAGE FR
\usepackage[cyr]{aeguill}
\usepackage[francais]{babel}
\usepackage{layout}
\usepackage[left=2.5cm, right=2.5cm]{geometry}
\usepackage{lipsum}
\usepackage{graphicx}
\graphicspath{{images/}}

% Don't forget to read the Memoir manual: memman.pdf

%%% Examples of Memoir customization
%%% enable, disable or adjust these as desired

%%% PAGE DIMENSIONS
% Set up the paper to be as close as possible to both A4 & letter:
\settrimmedsize{11in}{210mm}{*} % letter = 11in tall; a4 = 210mm wide
\setlength{\trimtop}{0pt}
\setlength{\trimedge}{\stockwidth}
\addtolength{\trimedge}{-\paperwidth}
%\settypeblocksize{*}{\lxvchars}{1.618} % we want to the text block to have golden proportionals
\setulmargins{40pt}{*}{*} % 50pt upper margins
%\setlrmargins{*}{*}{1.618} % golden ratio again for left/right margins
%\setheaderspaces{*}{*}{1.618}
%\checkandfixthelayout 
% This is from memman.pdf

%%% \maketitle CUSTOMISATION
% For more than trivial changes, you may as well do it yourself in a titlepage environment
\pretitle{\begin{center}\sffamily\huge\MakeUppercase}
\posttitle{\par\end{center}\vskip 0.5em}

%%% ToC (table of contents) APPEARANCE
\maxtocdepth{subsection} % include subsections
\renewcommand{\cftchapterpagefont}{}
\renewcommand{\cftchapterfont}{}     % no bold!

%%% HEADERS & FOOTERS
\pagestyle{ruled} % try also: empty , plain , headings , ruled , Ruled , companion

%%% CHAPTERS
\chapterstyle{hangnum} % try also: default , section , hangnum , companion , article, demo

\renewcommand{\chaptitlefont}{\Huge\sffamily\raggedright} % set sans serif chapter title font
\renewcommand{\chapnumfont}{\Huge\sffamily\raggedright} % set sans serif chapter number font

%%% SECTIONS
\hangsecnum % hang the section numbers into the margin to match \chapterstyle{hangnum}
\maxsecnumdepth{subsection} % number subsections

\setsecheadstyle{\Large\sffamily\raggedright} % set sans serif section font
\setsubsecheadstyle{\large\sffamily\raggedright} % set sans serif subsection font

%% END Memoir customization

\title{Master Degree Dissertation}
\author{Etudiant: Cyril Annette\\
Responsable: Pierre Bret\\
Entreprise: Smile Montpellier\\
Du 02 avril au 27 septembre 2013\\
}
\date{} % Delete this line to display the current date



%%% BEGIN DOCUMENT
\begin{document}


\maketitle
\begin{center}
	\includegraphics[width=300px]{smile.jpg}
	\includegraphics[width=300px]{supinfo.jpg}
\end{center}

\newpage
\tableofcontents* % the asterisk means that the contents itself isn't put into the ToC
\newpage

\chapter{Introduction}
\section{Présentation du candidat}

\paragraph{} \lipsum

\section{Presentation de l'entreprise}

\paragraph{} \lipsum

\subsection{test}

\paragraph{} \lipsum

\subsubsection{subtest}

\paragraph{} \lipsum

\newpage
\chapter{Projets}

\section{Analyse du contexte (RENAME ME) }

\paragraph{} \lipsum

\section{Problématiques (RENAME ME) }

\paragraph{} \lipsum

\section{Méthodes habituellement utilisées pour une situation présentant des similitudes (RENAME ME) }

\paragraph{} \lipsum

\newpage
\chapter{Prise de décisions}

\section{Exposé des décisions prises et des interventions menées par le stagiaire pour résoudre le problème (RENAME ME) }

\paragraph{} \lipsum

\section{Démonstration d’une originalité dans l’élaboration et la mise en œuvre de la solution (RENAME ME) }

\paragraph{} \lipsum

\section{ Analyse de l’approche choisie (RENAME ME) }

\paragraph{} \lipsum

\newpage
\chapter{Conclusions}

\section{ Réflexion sur le stage et le mémoire (RENAME ME) }

\paragraph{} \lipsum

\section{ Conclusion }

\paragraph{} \lipsum

\section{ Annexes }

\paragraph{} \lipsum

\end{document}
